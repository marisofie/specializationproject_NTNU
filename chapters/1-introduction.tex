\chapter{Introduction}
\label{chap:intro}
As the use of technological solutions is increasing, more and more data is generated each day. A special type of data is spatial data, which contains a geographical component. Spatial data is – amongst others – created by sensors tracking weather conditions, users tagging their location on social media and shipment companies tracking packages. As this type of data contains multiple dimensions, traditional index structures such as B-trees are not capable of storing the data in a sufficient manner. A solution to storing multidimensional data is \emph{R-trees}\cite{r-tree}. This index structure stores objects by encapsulating them in rectangles. By doing this, we can use multiple dimensions to identify an object instead of using a single index, such as in B-trees. \newline

\noindent
The general R-tree was created to insert incoming data one-by-one. This has however proven to be quite inefficient when there are large amounts of objects that need to be written to disk. There have been techniques developed to improve write performance in R-trees, such as bulk-inserting data\cite{SeededClustering}\cite{GBI}\cite{STLT}, but there are still other index structures that are created specifically to handle data loads with a high degree of write operations. One of these is the \emph{Log-Structured Merge-tree} (LSM-tree)\cite{LSMTree}. The LSM-tree handles incoming data in main memory, and performs the structuring of the data in components that are located in secondary memory. As a result, resources in main memory is only used to receive data objects, which improves the write performance.\newline

\noindent
An important part of the LSM-tree is the different merging policies. The policies define the number of components at each level of the tree. A high number of components usually lead to higher performance for write operations, as fewer entries have to be merged in each level. With a low number of components, query performance is improved as fewer elements has to be searched in order to find the wanted object. Different combinations of the merging policies is explored in Dostoevsky\cite{Dostoevsky}, and will be discussed further in the project.\newline

\noindent
Another significant aspect of LSM-trees is the need to impose an ordering on the data. The reason that ordering is important is because LSM-trees write the data sequentially. As R-trees handle multidimensional objects, they require an ordering scheme to sort the data by their location in space. A challenging part of ordering spatial data is that it can consist of many dimensions. It is hard to find an ordering that is valid across these dimensions, and it only gets more difficult as the number of dimensions increase. An important part of this project is therefore to explore different ordering schemes and find one that will be able to order the objects in a sufficient way.

\section{Project Goal}
The goal of this project is to create an index structure that is write-optimized and handles spatial data. In order to do this, the different aspects of LSM-trees and R-trees will be explored in order to create a design solution that incorporates R-trees into the secondary memory components in a LSM-tree. As there are many different parameters in LSM-trees and R-trees that can be adjusted, the main focus will be on merging policies and the creation and insertion into R-trees. 

\subsection{Research Questions}
\begin{itemize}
    \item Is it possible to implement an R-tree index structure into a LSM-tree?
    \item How can we perform merge-like operations to the R-trees in the LSM-tree?
\end{itemize}

\section{Project Structure}
This project is structured into three main parts: Theory, Related Work and Design Solution. In Theory the principles behind R-trees and LSM-trees are explored, in addition to different space-filling curves. Further, the different bulk-insertion and bulk-loading techniques for R-trees and the exploration of different merge policies done in Dostoevsky\cite{Dostoevsky} is looked at in Related Work. Finally, in Design Solution we look at a proposal for implementing the R-tree index structure into a LSM-tree. It also looks at how the design solution can be evaluated if implemented. 
