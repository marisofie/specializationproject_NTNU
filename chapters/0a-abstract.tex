\chapter*{Abstract}
The amount of spatial data generated is increasing every day. As geographical information is a part of the data generated from sensors, social media, shipment tracking and other data sources, there is a need for an index structure that handles a high write load where the data is multidimensional, such as in spatial data.\newline

\noindent
R-trees is an existing index structure that handles multidimensional data, while LSM-trees are created to specifically handle a high throughput of write operations. The different types of R-trees are discussed in detail, in addition to different bulk-loading and bulk-insertion methods. In addition, LSM-trees and its challenges are explained, such as write amplification and merge operations. Further, the ordering of data in R-trees are explored by examining space-filling curves and by looking at the dynamic Hilbert R-tree. Lastly, an exploration of different merge policies in LSM-trees is discussed.\newline

\noindent
A design solution for implementing the R-tree structure into a LSM-tree is also proposed. This design solution consists of taking advantage of the existing LSM-tree structure with levels, where the main-memory component only handles incoming data, while the lower levels structures the data by index. A merge policy similar to \emph{lazy leveling} is applied in order to minimize the need to merge data until it reaches the lowest level. By proposing this design solution we aim to create a write-optimized storage structure for multidimensional data. 
