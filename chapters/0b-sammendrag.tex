\chapter*{Sammendrag}

Mengden romlig data som blir generert øker for hver dag som går. Geografisk informasjon er en del av dataen som blir generert fra sensorer, sosiale medier, forsendelses-sporing og andre datakilder, og det er et behov for en indeksstruktur som håndterer en høy skrivelast hvor dataen er flerdimensjonal, slik som i romlig data.\newline

\noindent
R-trær er en eksisterende indeksstruktur som håndterer flerdimensjonal data, mens LSM-trær er laget for å spesifikt håndtere en høy mengde av skrive-operasjoner. De forskjellige typene R-trær er nøye diskutert, i tillegg til forskjellige bulklasting- og bulkinsetting-metoder. LSM-trær og tilhørende utfordringer er også diskutert, slikt som skrive-amplifikasjon og flette-operasjoner. Ordning av data i R-trær er utforsket ved å se på forskjellige romfyllingskurver og ved å se på dynamisk Hilbert R-tre. Til slutt blir forskjellige flette-metoder i LSM-trær diskutert.\newline

\noindent
En designløsning for å implementere strukturen i et R-tre inn i et LSM-tre blir også foreslått. Denne løsningen innebærer å utnytte den eksisterende strukturen i et LSM-tre med nivåer, hvor komponenten i hovedminne kun håndterer innkommende data, mens de lavere nivåene strukturerer dataen ved indeksering. En flette-metode lik \emph{lazy leveling} blir anvendt for å minimere behovet for fletting av data i de øverste nivåene. Ved å foreslå denne designløsningen ønsker vi å lage en skrive-optimisert lagringsstruktur for flerdimensjonal data. 
